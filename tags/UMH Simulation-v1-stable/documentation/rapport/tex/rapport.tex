\documentclass[a4paper,11pt]{article}
\usepackage{configuration}


\begin{document}
%\begin{titlepage}
 \thispagestyle{empty}
\begin{center}
 
\textsc{\LARGE Université de Mons-Hainaut}\\[0.5cm]
\textsc{\Large Faculté des Sciences}\\[4.0cm]
\textsc{\Large Simulation de systèmes à évènements discrets}\\[0.5cm]
 
 
% titre
\HRule \\[0.4cm]
{ \Large \bfseries Rapport du projet}\\[0.3cm]
 
\HRule \\[1.5cm]
 
% auteur et directeur
\begin{minipage}{0.4\textwidth}
\begin{flushleft} \large
\emph{Auteurs:} 
\\ Sébastien \textsc{Dubois}
\\ Jean-François \textsc{Mernier}
\\ Frédéric \textsc{Regnier}

\end{flushleft}
\end{minipage}
\begin{minipage}{0.4\textwidth}
\begin{flushright} \large
%\emph{Directeur:} \\ Olivier \textsc{Markowitch}
\end{flushright}
\end{minipage}


% remplit la page (de vide :p)
\vfill

% date avec un peu d'espace ensuite
{\Large \today}\\[1.5cm]

% logos en bas de page
\includegraphics[scale=0.03]{umh/logo-acwalbxl}
\includegraphics[scale=0.08]{umh/logo-umh}

\end{center}


% ajoute une page vierge après la page de garde
\newpage
\thispagestyle{empty}
\mbox{}
\newpage
%\end{titlepage}

\tableofcontents % table des matières
\pagebreak
\listoffigures % liste des figures
\pagebreak
%\listoftables % liste des tables
%\pagebreak


\section{Evenements}

\subsection{Hotes}
\subsubsection{Envoi d'un message original}
\subsubsection{Réception d'un message}
\subsubsection{Fin de traitement d'un message}
\subsubsection{Timeout}

\subsection{Agents}
\subsubsection{Réception d'un message}
\subsubsection{Fin de traitement d'un message}
\subsubsection{Envoi des informations de routage}
\subsubsection{Réception d'informations de routage}


\section{Résultats}
\subsection{Paramètres du système}

\subsubsection{Hote}
\begin{itemize}
 \item Durée du timeout (temps après lequel on réémet un message)
 \item Temps maximal inter-envois (pour les messages originaux)
 \item Temps de traitement d'un message
 \item Pourcentage de messages à destination d'un autre agent
\end{itemize}


\subsubsection{Agent}

\begin{itemize}
 \item Nombre d'hôtes reliés
 \item Taux de pertes brutales de messages
 \item Temps de traitement d'un message
 \item Taille de buffer (en entrée)
\end{itemize}

Pour le distance vector on a en plus:
\begin{itemize}
 \item Temps inter-envois des informations de routage
\end{itemize}

\subsubsection{Simulation}
\begin{itemize}
 \item Durée
 \item Délai agent <-> hôte
 \item Distance vector activé (oui/non)
 \item Durée de la période d'initialisation
 \item Périodicité d'affichage des statistiques (e.g., tous les 1\% de simulation)
\end{itemize}



\section{Décisions}

\subsection{Gestion des évènements}
Nous avons choisi d'utiliser une seule FEL pour la simulation. On y place tous les évènements. De plus, pour un temps $t$ de simulation donné, nous avons décidé de traiter certains évènements prioritairement.

\begin{itemize}
 \item En premier lieu on traite les évènements de réception d'informations de routage
 \item Ensuite on traite les évènements de réception d'informations de routage
 \item Ensuite on traite les évènements de réception de messages (accusés et messages normaux)
 \item Ensuite on traite les évènements de timeout (un message pour lequel on a pas encore reçu d'accusé)
\end{itemize}

TODO ajouter explication pour la fin de simulation quand on aura pris une décision!


\subsection{Distance vector}
Nous avons choisi de modifier les coûts en fonction du taux d'occupation du buffer de l'agent. Pour ce faire

TODO continuer


%\pagebreak
\clearpage

%\backmatter % pour ne pas numéroter à la fin
\appendix

\section{Le programme et son utilisation}
TODO
\appendix



\end{document}
